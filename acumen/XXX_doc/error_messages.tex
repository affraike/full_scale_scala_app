\section{List of error messages}
\subsubsection{Interpreter Errors}

The following errors are raised by the interpreter. This means they can occur both in the command-line and the graphical versions of Acumen.
\subsubsection{Parse Error}

Example:

\begin{lstlisting}
[2.1] failure: ``class'' expected but identifier cass found

cass B() end
^
\end{lstlisting}

Here, [2.1] reads "line 2, column 1".
\subsubsection{Found no instance of class classname.}

Example:

\begin{lstlisting}
Found no instance of class Main.
\end{lstlisting}

Should never happen in a regular Acumen program written using the GUI.
\subsubsection{Class classname is not defined.}

A program tries to create a class that has not been defined.
\subsubsection{value is not an object.}

For instance, the program
\begin{lstlisting}
...
(1+2).x = 4
...
\end{lstlisting}
%
will raise
%
\begin{lstlisting}
3 is not an object
\end{lstlisting}

\subsubsection{Cannot convert value into a type.}

For instance, the program
%
\begin{lstlisting}
...
"a" + 1.5
...
\end{lstlisting}
%
will raise
%
\begin{lstlisting}
Cannot convert "a" into a float.
\end{lstlisting}

\subsubsection{value is not a vector or a list.}

For instance, the program
%
\begin{lstlisting}
...
for i=1+2 
  x = x+i
end
...
\end{lstlisting}
%
will raise
%
\begin{lstlisting}
3 is not a vector or a list.
\end{lstlisting}

\subsubsection{Unknown operator name}

An undefined function has been invoked. For instance:
%
\begin{lstlisting}
foo(1,2)
\end{lstlisting}
%
as opposed to
%
\begin{lstlisting}
sqrt(2)
\end{lstlisting}
%
which is defined.
\subsubsection{Cross product only defined over vectors of size 3.}

When trying to perform a cross product over vector the length of which is not 3.
\subsubsection{name is not a valid vector-vector operation.}

For instance, the program
%
\begin{lstlisting}
v = 1:10;
w = 1:20;
res = v / w
\end{lstlisting}
%
is not valid since \lstinline{/} is not defined over vectors.
\subsubsection{name is not a valid operation over vectors.}
%
For instance, the program
%
\begin{lstlisting}
sqrt(1:10)
\end{lstlisting}
%
is not valid since \lstinline{sqrt} is not defined over lists.
\subsubsection{name is not a valid operation over lists.}

For instance, the program
%
\begin{lstlisting}
sqrt(self.children)
\end{lstlisting}
%
is not valid since sqrt is not defined over vectors.
\subsubsection{name is not a valid scalar-vector operation.}

For instance, the program
%
\begin{lstlisting}
2.5 .* (1:10)
\end{lstlisting}
%
will raise
%
\begin{lstlisting}
.* is not a valid scalar-vector operation.
\end{lstlisting}

\subsubsection{name is not a valid vector-scalar operation.}

For instance, the program
%
\begin{lstlisting}
(1:10) .* 2.5
\end{lstlisting}
%
will raise
%
\begin{lstlisting}
.* is not a valid vector-scalar operation.
\end{lstlisting}

\subsubsection{Too many arguments in the construction of classname.}

This error occurs when too many arguments are passed to a class constructor during creation.
\subsubsection{Not enough arguments in the construction of classname.}

This error occurs when too many arguments are passed to a class constructor during creation.
\subsubsection{Variable name not declared}

This happens when a program try to read or write a variable that has not been declared (either as a class parameter or as a private variable).
\subsubsection{Object id is not self (i.e. id) nor a child of self (i.e. id1, id2, ...).}

This message is raised whenever a program tries to access an object's field, the latter object being neither self nor a direct child of self.
\subsubsection{Object id is not a child of id.}

This happens during a "bad move":
%
\begin{lstlisting}
...
move o1.x o2
...
\end{lstlisting}
%
would raise this error whenever the object designated by o1.x is not a child of o1.
\subsubsection{No case matching value.}

For instance, the program
%
\begin{lstlisting}
...
switch (1+2)
  case 1
    ...
  case 2
    ...
end
...
\end{lstlisting}
%
will raise
%
\begin{lstlisting}
No case matching 3.
\end{lstlisting}

\subsubsection{The left hand-side of an assignment must be of the form 'e.x'.}

This error happens when the interpreter encounters a statement of the form
%
\begin{lstlisting}
x = 2
\end{lstlisting}
%
instead of
%
\begin{lstlisting}
o.x = 2
\end{lstlisting}

Since Acumen inserts self whenever needed, this should never happen in a regular Acumen program.
\subsubsection{The left hand-side of an equation must be a field.}

This may happen with a program like
%
\begin{lstlisting}
1 = 2
\end{lstlisting}

However, this should never happen in a regular Acumen program, since a error saying is not of the form e.x should be raised beforehand.
\subsubsection{Move statements must have the form 'move o1.x o2'}

For instance, the program
%
\begin{lstlisting}
...
move x o2 
...
\end{lstlisting}
%
will raise such an error. Since Acumen inserts self whenever needed, this should never happen in a regular Acumen program.
\subsubsection{GUI Errors}

The following errors may be raised by the graphical user interface only.
\subsubsection{Simulation's time is not a double.}

This error occurs when the simulation's time has been set to a string for example.
\subsubsection{Simulation's stepType is not a step type.}

This error occurs when the simulation's stepType has been set to a string for example.
\subsubsection{Can only state when state or state}

For instance

\begin{lstlisting}
Can only play when Paused or Stopped.
\end{lstlisting}

This should normally never happen since the GUI prevents the user from clicking on "play" when the simulation is already running.
\subsubsection{Command-Line errors}

The following errors may be raised by the command-line version only.
\subsubsection{Bad program options. Valid options are ...}

This may be raised by
%
\begin{lstlisting}
./acumen foo 
\end{lstlisting}
%
for instance, since foo is not a valid option.
\subsubsection{Special Errors}

The following errors are very special.
\subsubsection{The "impossible" has just happened! ...}

As the rest of the message states, if this occurs, please report a bug at \url{http://code.google.com/p/acumen-language/issues/} including the program that has led to that error. 
